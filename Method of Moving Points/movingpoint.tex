%%%%%%%%%%%%%%%%%%%%%%%%%%%%% Define Article %%%%%%%%%%%%%%%%%%%%%%%%%%%%%%%%%%
\documentclass{article}
%%%%%%%%%%%%%%%%%%%%%%%%%%%%%%%%%%%%%%%%%%%%%%%%%%%%%%%%%%%%%%%%%%%%%%%%%%%%%%%

%%%%%%%%%%%%%%%%%%%%%%%%%%%%% Using Packages %%%%%%%%%%%%%%%%%%%%%%%%%%%%%%%%%%
\usepackage{geometry}
\usepackage{graphicx}
\usepackage{amssymb}
\usepackage{amsmath}
\usepackage{amsthm}
\usepackage{empheq}
\usepackage{mdframed}
\usepackage{booktabs}
\usepackage{lipsum}
\usepackage{graphicx}
\usepackage{color}
\usepackage{psfrag}
\usepackage{pgfplots}
\usepackage{bm}
\usepackage[colorlinks=true, urlcolor=blue, linkcolor=red]{hyperref}
%%%%%%%%%%%%%%%%%%%%%%%%%%%%%%%%%%%%%%%%%%%%%%%%%%%%%%%%%%%%%%%%%%%%%%%%%%%%%%%

% Other Settings

%%%%%%%%%%%%%%%%%%%%%%%%%% Page Setting %%%%%%%%%%%%%%%%%%%%%%%%%%%%%%%%%%%%%%%
\geometry{a4paper}

%%%%%%%%%%%%%%%%%%%%%%%%%% Define some useful colors %%%%%%%%%%%%%%%%%%%%%%%%%%
\definecolor{ocre}{RGB}{243,102,25}
\definecolor{mygray}{RGB}{243,243,244}
\definecolor{deepGreen}{RGB}{26,111,0}
\definecolor{shallowGreen}{RGB}{235,255,255}
\definecolor{deepBlue}{RGB}{61,124,222}
\definecolor{shallowBlue}{RGB}{235,249,255}
%%%%%%%%%%%%%%%%%%%%%%%%%%%%%%%%%%%%%%%%%%%%%%%%%%%%%%%%%%%%%%%%%%%%%%%%%%%%%%%

%%%%%%%%%%%%%%%%%%%%%%%%%% Define an orangebox command %%%%%%%%%%%%%%%%%%%%%%%%
\newcommand\orangebox[1]{\fcolorbox{ocre}{mygray}{\hspace{1em}#1\hspace{1em}}}
%%%%%%%%%%%%%%%%%%%%%%%%%%%%%%%%%%%%%%%%%%%%%%%%%%%%%%%%%%%%%%%%%%%%%%%%%%%%%%%

%%%%%%%%%%%%%%%%%%%%%%%%%%%% English Environments %%%%%%%%%%%%%%%%%%%%%%%%%%%%%
\newtheoremstyle{mytheoremstyle}{3pt}{3pt}{\normalfont}{0cm}{\rmfamily\bfseries}{}{1em}{{\color{black}\thmname{#1}~\thmnumber{#2}}\thmnote{\,--\,#3}}
\newtheoremstyle{myproblemstyle}{3pt}{3pt}{\normalfont}{0cm}{\rmfamily\bfseries}{}{1em}{{\color{black}\thmname{#1}~\thmnumber{#2}}\thmnote{\,--\,#3}}
\theoremstyle{mytheoremstyle}
\newmdtheoremenv[linewidth=1pt,backgroundcolor=shallowGreen,linecolor=deepGreen,leftmargin=0pt,innerleftmargin=20pt,innerrightmargin=20pt,]{theorem}{Theorem}[section]
\theoremstyle{mytheoremstyle}
\newmdtheoremenv[linewidth=1pt,backgroundcolor=shallowBlue,linecolor=deepBlue,leftmargin=0pt,innerleftmargin=20pt,innerrightmargin=20pt,]{definition}{Definition}[section]
\theoremstyle{myproblemstyle}
\newmdtheoremenv[linecolor=black,leftmargin=0pt,innerleftmargin=10pt,innerrightmargin=10pt,]{problem}{Problem}[section]
%%%%%%%%%%%%%%%%%%%%%%%%%%%%%%%%%%%%%%%%%%%%%%%%%%%%%%%%%%%%%%%%%%%%%%%%%%%%%%%

%%%%%%%%%%%%%%%%%%%%%%%%%%%%%%% Plotting Settings %%%%%%%%%%%%%%%%%%%%%%%%%%%%%
\usepgfplotslibrary{colorbrewer}
\pgfplotsset{width=8cm,compat=1.9}
%%%%%%%%%%%%%%%%%%%%%%%%%%%%%%%%%%%%%%%%%%%%%%%%%%%%%%%%%%%%%%%%%%%%%%%%%%%%%%%

%%%%%%%%%%%%%%%%%%%%%%%%%%%%%%% Title & Author %%%%%%%%%%%%%%%%%%%%%%%%%%%%%%%%
\title{Method of Moving Points}
\author{Alston Yam}
\date{}
\parskip=5pt
\parindent=0pt
%%%%%%%%%%%%%%%%%%%%%%%%%%%%%%%%%%%%%%%%%%%%%%%%%%%%%%%%%%%%%%%%%%%%%%%%%%%%%%%

\begin{document}
    \maketitle

    \section{Introduction}
    Inspired by \href{https://web.evanchen.cc/static/mop/mockimo/2020.pdf}{problem 4} on this IMO mock by Evan Chen.

    \section{Cross Ratios}
    \begin{definition}[Cross Ratios]
        Given 4 distinct points $A, B, C, D$ on a line, the cross ratio $(A, B; C, D)$ is defined as \[(A, B; C, D) = \frac{AC \cdot BD}{BC \cdot AD}\]
        Where the lengths are taken to be directed $(XY = -YX)$.
    \end{definition}

    We can actually extend the definition of the cross ratio to not just points on a line, but also four points on a conic $\gamma$ (the most commonly used conic in Olympiad geometry is a circle), and also a \textit{pencil} of lines through a particular point. In the latter case $A, B, C, D$ will correspond to lines rather than points.

    In the case of a pencil, the cross ratio can actually be thought of as the ratio of the sines of the angles between these four lines. 

    \section{Projective Transformations}

    A \textit{projective transformation} is any transformation that preserves the cross ratio. Specifically:

    \begin{definition}[Projective Transformations]
        A projective map $f$ is defined as a function $f:\mathcal{C}_1 \to \mathcal{C}_2$ (where $\mathcal{C}_1$ and $\mathcal{C}_2$ are both conics, lines or pencils of lines) such that for any 4 points $A, B, C, D \in \mathcal{C}_1$, \[(A, B; C, D) = (f(A), f(B); f(C), f(D))\]
    \end{definition}

    There are two useful results that would come in handy.

    \begin{theorem}[Projective Compositions]
        The composition $f \circ g$ of two projective functions $f$ and $g$ is projective.
    \end{theorem}

    \begin{theorem}[Inverse of a Projective Map]
        The inverse $f^{-1}$ of a projective map $f$ is also projective.
    \end{theorem}

    

    We give a few examples of common projective transformations below. These are taken from \href{https://artofproblemsolving.com/community/c473124h1763266_moving_points_tutorial}{this blog post}.


    \subsection{Common Projective Transformations}
    \subsubsection{Projection from a line to a pencil of lines}
    \subsubsection{Projection from a line to another line}
    \subsubsection{Reflection across a line}
    \subsubsection{Projection from a conic to a pencil of lines}
    \subsubsection{Projection from a conic to points on that same conic}
    \subsubsection{Inversion}

    \section{The Method}
    The essence of the method of moving points boils down to one important theorem:
    \begin{theorem}
        hello
    \end{theorem}

    
\end{document}