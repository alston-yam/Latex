%%%%%%%%%%%%%%%%%%%%%%%%%%%%% Define Article %%%%%%%%%%%%%%%%%%%%%%%%%%%%%%%%%%
\documentclass{article}
%%%%%%%%%%%%%%%%%%%%%%%%%%%%%%%%%%%%%%%%%%%%%%%%%%%%%%%%%%%%%%%%%%%%%%%%%%%%%%%

%%%%%%%%%%%%%%%%%%%%%%%%%%%%% Using Packages %%%%%%%%%%%%%%%%%%%%%%%%%%%%%%%%%%
\usepackage{geometry}
\usepackage{graphicx}
\usepackage{amssymb}
\usepackage{amsmath}
\usepackage{amsthm}
\usepackage{empheq}
\usepackage{mdframed}
\usepackage{booktabs}
\usepackage{lipsum}
\usepackage{graphicx}
\usepackage{color}
\usepackage{psfrag}
\usepackage{pgfplots}
\usepackage{bm}
\usepackage{dirtytalk}
%%%%%%%%%%%%%%%%%%%%%%%%%%%%%%%%%%%%%%%%%%%%%%%%%%%%%%%%%%%%%%%%%%%%%%%%%%%%%%%

% Other Settings

%%%%%%%%%%%%%%%%%%%%%%%%%% Page Setting %%%%%%%%%%%%%%%%%%%%%%%%%%%%%%%%%%%%%%%
\geometry{a4paper}

%%%%%%%%%%%%%%%%%%%%%%%%%% Define some useful colors %%%%%%%%%%%%%%%%%%%%%%%%%%
\definecolor{ocre}{RGB}{243,102,25}
\definecolor{mygray}{RGB}{243,243,244}
\definecolor{deepGreen}{RGB}{26,111,0}
\definecolor{shallowGreen}{RGB}{235,255,255}
\definecolor{deepBlue}{RGB}{61,124,222}
\definecolor{shallowBlue}{RGB}{235,249,255}
%%%%%%%%%%%%%%%%%%%%%%%%%%%%%%%%%%%%%%%%%%%%%%%%%%%%%%%%%%%%%%%%%%%%%%%%%%%%%%%

%%%%%%%%%%%%%%%%%%%%%%%%%% Define an orangebox command %%%%%%%%%%%%%%%%%%%%%%%%
\newcommand\orangebox[1]{\fcolorbox{ocre}{mygray}{\hspace{1em}#1\hspace{1em}}}
%%%%%%%%%%%%%%%%%%%%%%%%%%%%%%%%%%%%%%%%%%%%%%%%%%%%%%%%%%%%%%%%%%%%%%%%%%%%%%%

%%%%%%%%%%%%%%%%%%%%%%%%%%%% English Environments %%%%%%%%%%%%%%%%%%%%%%%%%%%%%
\newtheoremstyle{mytheoremstyle}{3pt}{3pt}{\normalfont}{0cm}{\rmfamily\bfseries}{}{1em}{{\color{black}\thmname{#1}~\thmnumber{#2}}\thmnote{\,--\,#3}}
\newtheoremstyle{myproblemstyle}{3pt}{3pt}{\normalfont}{0cm}{\rmfamily\bfseries}{}{1em}{{\color{black}\thmname{#1}~\thmnumber{#2}}\thmnote{\,--\,#3}}
\theoremstyle{mytheoremstyle}
\newmdtheoremenv[linewidth=1pt,backgroundcolor=shallowGreen,linecolor=deepGreen,leftmargin=0pt,innerleftmargin=20pt,innerrightmargin=20pt,]{theorem}{Theorem}[section]
\theoremstyle{mytheoremstyle}
\newmdtheoremenv[linewidth=1pt,backgroundcolor=shallowBlue,linecolor=deepBlue,leftmargin=0pt,innerleftmargin=20pt,innerrightmargin=20pt,]{definition}{Definition}[section]
\theoremstyle{myproblemstyle}
\newmdtheoremenv[linecolor=black,leftmargin=0pt,innerleftmargin=10pt,innerrightmargin=10pt,]{problem}{Problem}[section]
%%%%%%%%%%%%%%%%%%%%%%%%%%%%%%%%%%%%%%%%%%%%%%%%%%%%%%%%%%%%%%%%%%%%%%%%%%%%%%%

%%%%%%%%%%%%%%%%%%%%%%%%%%%%%%% Plotting Settings %%%%%%%%%%%%%%%%%%%%%%%%%%%%%
\usepgfplotslibrary{colorbrewer}
\pgfplotsset{width=8cm,compat=1.9}
\parskip=5pt
\parindent=0in
%%%%%%%%%%%%%%%%%%%%%%%%%%%%%%%%%%%%%%%%%%%%%%%%%%%%%%%%%%%%%%%%%%%%%%%%%%%%%%%

%%%%%%%%%%%%%%%%%%%%%%%%%%%%%%% Title & Author %%%%%%%%%%%%%%%%%%%%%%%%%%%%%%%%
\title{Fermat's Little Theorem and Euler's Theorem}
\author{Alston Yam}
\date{}
%%%%%%%%%%%%%%%%%%%%%%%%%%%%%%%%%%%%%%%%%%%%%%%%%%%%%%%%%%%%%%%%%%%%%%%%%%%%%%%

\begin{document}
    \maketitle

    \section{Fermat's Little Theorem}
    
    \begin{theorem}[Fermat's Little Theorem (FLT)]
        Let $p$ be a prime, and let $a$ be any positive integer not divisible by $p$. Then \[a^{p-1} \equiv 1 \pmod{p}\]    
    \end{theorem}

    There's also an alternate formulation: 

    \begin{theorem}[Alternate form of FLT]
        Let $p$ be a prime and let $a$ be any positive integer. Then we have
        \[a^p \equiv a \pmod{p}\]
    \end{theorem}

    We can see that to go from the first formulation to the second, we have essentially multiplied both sides by $a$. However to go from the second formulation back to the first, we must introduce the new condition that $\gcd(a, p) = 1$. This is because of the edge case of $p \mid a$.

    \begin{center}
        \say{If something is worth proving once, it's worth proving twice.}
    \end{center}

    Therefore, we will present two different proofs of FLT.

    
    \textbf{Proof 1: Induction}
    \begin{proof}
        We will use a key lemma:

        {\color{red}{\textbf{Lemma 1:}}} $(a + b)^p \equiv a^p + b^p \pmod{p}$ (\textit{Side note: this lemma is called ``Freshman's Dream''}).
        {\color{pink}{\textbf{Proof of Lemma:}}}
        \begin{center}
            \begin{align*}
                (a + b)^p &= a^p + {{p}\choose{1}}a^{p-1}b + {{p}\choose{2}}a^{p-2}b^{2} + \cdots + {{p}\choose{p-1}}ab^{p-1} + b^{p}\\
                &\equiv a^p + b^p \pmod{p}
            \end{align*}
        \end{center}

        This is because all of the ``middle'' terms have a binomial coefficient, and we know that $p \mid {{p}\choose{i}}$ for all $1 \leq i \leq p-1$. Therefore, all of the middle terms disappear when we take mod $p$, and the lemma is proven.

        I claim $a^p \equiv a \pmod{p}$ for all integers $a$ and prime $p$. We induct on $a$.

        \textit{Base case:} $a = 1$
        \[1^p \equiv 1 \pmod{p}\]
        And we know this will hold true for any prime $p$.

        \textit{Inductive Hypothesis:} Assume that for $k = a$, we have $k^p \equiv k \pmod{p}$ for all primes $p$.

        \textit{Inductive Step:} Now we will consider $k+1$. Notice that:
        \begin{center}
            \begin{align*}
                (k+1)^p &\equiv k^p + 1^p \pmod{p}\\
                &\equiv k + 1 \pmod{p}
            \end{align*}
        \end{center}

        Where the first line is from Lemma 1 and the second line is from the Inductive Hypothesis. Hence our induction is complete, and we have proven FLT.
    \end{proof}

    \textbf{Proof 2: Inverses}
    \begin{proof}
        Here I will claim that $a^{p-1} \equiv 1 \pmod{p}$, for  positive integers $a$ that are coprime with $p$.

        Let's consider the sets $\{1, 2, 3, \hdots, p-2, p-1\}$, and $\{a, 2a, 3a, \hdots, (p-2)a, (p-1)a\}$. The main claim is that the two sets are \textbf{permutations} of one another under mod $p$. Notice that the elements in the first set are all distinct under mod $p$, and since $a$ is coprime to $p$, we know that all the elements in the second set are not multiples of $p$. Therefore if we prove all elements in the second set are distinct under mod $p$, we would have proven the claim.

        Suppose not. Then, we must have two elements such that $ia \equiv ja \pmod{p}$. Since $a$ is coprime to $p$, we may multiply both sides of the congruence by $a^{-1}$. This implies $i \equiv j \pmod{p}$ which is a contradiction. 

        So now we know the second set is a permutation of the first, under mod $p$. We will multiply all the elements in both sets together which implies
        \[\prod_{k = 1}^{p-1}(k) \equiv a^{p-1}\prod_{k = 1}^{p-1}(k) \pmod{p}\]

        However, $\gcd\left(p, \prod_{k = 1}^{p-1}(k)\right) = 1$, so we can may multiply both sides by the inverse of $\prod_{k = 1}^{p-1}(k)$ to imply \[a^{a-1} \equiv 1 \pmod{p}\] as desired. 
    \end{proof}

    There is also a cool combinatorial proof that does not require any words, which you are welcome to look up.

    \section{Euler's Theorem}
    To recap, we define the Euler Totient Function again:
    \begin{definition}[Euler's Totient Function]
        $\varphi(n)$ is defined as the number of positive integers $k$ with $k \leq n$ such that $\gcd{k, n} = 1$.
    \end{definition}

    The most important result relating to this function is Euler's Theorem:
    \begin{theorem}[Euler's Theorem]
        For any positive integers $n \geq 2$ and an integer $a$ coprime to $n$, we have \[a^{\varphi(n)} \equiv 1 \pmod{n}\]        
    \end{theorem}

    The statement of Euler's Theorem closely resembles FLT, and we might notice that this theorem is actually a generalisation of it. In fact, this lets us extend our computation from a prime mod to any general positive integer mod.

    \textbf{Proof of Euler's Theorem}

    \begin{proof}
        The proof of Euler's Theorem is extremely similar to the proof by inverses for FLT, and you're encouraged to ponder it before reading on. For the sake of completeness, I will present a proof here.
        
        Consider the set $S = \{k \mid \gcd(k, n) = 1\}$. We know $|S| = \varphi(n)$ by definition. Also consider the set $T = \{ak \mid \gcd(k, n) = 1\}$. We know that all elements in both $S$ ad $K$ are coprime to $n$, and we know that $K$ is a permutation of $S$ under mod $n$ (this can be proven in exactly the same way as the proof by inverse previously).

        Now, we also multiply all elements in both sets together and equate them. 
        \[a^{|S|} \prod_{1 \leq k \leq n, \gcd(k, n) = 1}^{}(k) \equiv \prod_{1 \leq k \leq n, \gcd(k, n) = 1}^{}(k) \pmod{p} \]

        Then  we will multiply both sides by the inverse of the large producted term, to arrive at $a^{\varphi(n)} \equiv 1 \pmod{n}$, as desired.
    \end{proof}

    \section{Problems}
    \begin{problem}
        Find
        \[2^{50} \pmod{7}\]
    \end{problem}
    
    \begin{problem}
        Let $a, b$ by integers and $p$ a prime. Show that $p$ divides $ab^p - a^pb$.
    \end{problem}

    \begin{problem}
        Show that $n \mid 2^{n!} - 1$ for all odd integers $n$.
    \end{problem}

    \begin{problem}
        A positive integer $n$ is called \textit{groovy} if, for every positive integer $a$, $n^2$ divides $a^n - 1$ whenever $n$ divides $a^n - 1$.

        Show that all primes are \textit{groovy}.
    \end{problem}

    \begin{problem}
        Show that for every positive integer $n$ not divisible by $2$ or $5$, there exists a multiple of $n$ all of whose digits are ones.
    \end{problem}

    
    
\end{document}