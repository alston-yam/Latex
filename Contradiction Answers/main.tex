%%%%%%%%%%%%%%%%%%%%%%%%%%%%% Define Article %%%%%%%%%%%%%%%%%%%%%%%%%%%%%%%%%%
\documentclass{article}
%%%%%%%%%%%%%%%%%%%%%%%%%%%%%%%%%%%%%%%%%%%%%%%%%%%%%%%%%%%%%%%%%%%%%%%%%%%%%%%

%%%%%%%%%%%%%%%%%%%%%%%%%%%%% Using Packages %%%%%%%%%%%%%%%%%%%%%%%%%%%%%%%%%%
\usepackage{geometry}
\usepackage{graphicx}
\usepackage{amssymb}
\usepackage{amsmath}
\usepackage{amsthm}
\usepackage{empheq}
\usepackage{mdframed}
\usepackage{booktabs}
\usepackage{lipsum}
\usepackage{graphicx}
\usepackage{color}
\usepackage{psfrag}
\usepackage{pgfplots}
\usepackage{asymptote}
\usepackage{bm}
%%%%%%%%%%%%%%%%%%%%%%%%%%%%%%%%%%%%%%%%%%%%%%%%%%%%%%%%%%%%%%%%%%%%%%%%%%%%%%%

% Other Settings

%%%%%%%%%%%%%%%%%%%%%%%%%% Page Setting %%%%%%%%%%%%%%%%%%%%%%%%%%%%%%%%%%%%%%%
\geometry{a4paper}

%%%%%%%%%%%%%%%%%%%%%%%%%% Define some useful colors %%%%%%%%%%%%%%%%%%%%%%%%%%
\definecolor{ocre}{RGB}{243,102,25}
\definecolor{mygray}{RGB}{243,243,244}
\definecolor{deepGreen}{RGB}{26,111,0}
\definecolor{shallowGreen}{RGB}{235,255,255}
\definecolor{deepBlue}{RGB}{61,124,222}
\definecolor{shallowBlue}{RGB}{235,249,255}
%%%%%%%%%%%%%%%%%%%%%%%%%%%%%%%%%%%%%%%%%%%%%%%%%%%%%%%%%%%%%%%%%%%%%%%%%%%%%%%

%%%%%%%%%%%%%%%%%%%%%%%%%% Define an orangebox command %%%%%%%%%%%%%%%%%%%%%%%%
\newcommand\orangebox[1]{\fcolorbox{ocre}{mygray}{\hspace{1em}#1\hspace{1em}}}
%%%%%%%%%%%%%%%%%%%%%%%%%%%%%%%%%%%%%%%%%%%%%%%%%%%%%%%%%%%%%%%%%%%%%%%%%%%%%%%

%%%%%%%%%%%%%%%%%%%%%%%%%%%% English Environments %%%%%%%%%%%%%%%%%%%%%%%%%%%%%
\newtheoremstyle{mytheoremstyle}{3pt}{3pt}{\normalfont}{0cm}{\rmfamily\bfseries}{}{1em}{{\color{black}\thmname{#1}~\thmnumber{#2}}\thmnote{\,--\,#3}}
\newtheoremstyle{myproblemstyle}{3pt}{3pt}{\normalfont}{0cm}{\rmfamily\bfseries}{}{1em}{{\color{black}\thmname{#1}~\thmnumber{#2}}\thmnote{\,--\,#3}}
\theoremstyle{mytheoremstyle}
\newmdtheoremenv[linewidth=1pt,backgroundcolor=shallowGreen,linecolor=deepGreen,leftmargin=0pt,innerleftmargin=20pt,innerrightmargin=20pt,]{theorem}{Theorem}[section]
\theoremstyle{mytheoremstyle}
\newmdtheoremenv[linewidth=1pt,backgroundcolor=shallowBlue,linecolor=deepBlue,leftmargin=0pt,innerleftmargin=20pt,innerrightmargin=20pt,]{definition}{Definition}[section]
\theoremstyle{myproblemstyle}
\newmdtheoremenv[linecolor=black,leftmargin=0pt,innerleftmargin=10pt,innerrightmargin=10pt,]{problem}{Problem}[section]
%%%%%%%%%%%%%%%%%%%%%%%%%%%%%%%%%%%%%%%%%%%%%%%%%%%%%%%%%%%%%%%%%%%%%%%%%%%%%%%

%%%%%%%%%%%%%%%%%%%%%%%%%%%%%%% Plotting Settings %%%%%%%%%%%%%%%%%%%%%%%%%%%%%
\usepgfplotslibrary{colorbrewer}
\pgfplotsset{width=8cm,compat=1.9}
\parindent=0in
%%%%%%%%%%%%%%%%%%%%%%%%%%%%%%%%%%%%%%%%%%%%%%%%%%%%%%%%%%%%%%%%%%%%%%%%%%%%%%%

%%%%%%%%%%%%%%%%%%%%%%%%%%%%%%% Title & Author %%%%%%%%%%%%%%%%%%%%%%%%%%%%%%%%
\title{Contradiction Answers}
\author{Alston Yam}
%%%%%%%%%%%%%%%%%%%%%%%%%%%%%%%%%%%%%%%%%%%%%%%%%%%%%%%%%%%%%%%%%%%%%%%%%%%%%%%

\begin{document}
    \maketitle
    \section{Solutions}
    \begin{problem}
        Prove that $\sqrt{2}$ is irrational.
    \end{problem}

    \begin{proof}
        \textit{FTSOC,} Suppose $\sqrt{2}$ is rational. Then, $\sqrt{2} = \frac{a}{b}$ for some integers $a$ and $b$ with no common factors. Squaring both sides, we get $2 = \frac{a^2}{b^2}$. Thus, $2b^2 = a^2$. Since $a^2$ is even, $a$ must be even. Let $a = 2k$ for some integer $k$. Then, $2b^2 = {(2k)}^2 = 4k^2$. Thus, $b^2 = 2k^2$. Since $b^2$ is even, $b$ must be even. However, this contradicts the fact that $a$ and $b$ have no common factors. Thus, $\sqrt{2}$ is irrational.
    \end{proof}

    \begin{problem}
        Prove that $\sqrt[m]{n}$ is either irrational or an integer $\forall n, m \in \mathbb{N}$.
    \end{problem}

    \begin{proof}
        \textit{FTSOC,} Suppose $\sqrt[m]{n}$ is neither irrational nor an integer. Then, $\sqrt[m]{n} = \frac{a}{b}$ for some integers $a$ and $b$ with no common factors. Raising both sides to the $m$th power, we get $n = \frac{a^m}{b^m}$. Thus, $nb^m = a^m$. However, notice that since $a$ is an integer, if the RHS is a perfect $m$th power, then the LHS must also be an $m$th power. This implies that $n$ is an $m$th power, i.e. $n = k^m$ for some integer $k$. This is a contradiction as $\sqrt[m]{n} = k$ which is an integer, Contradiction!
    \end{proof}

    \begin{problem}
        Show that if $2^x = 3$, then $x$ is irrational.
    \end{problem}

    \begin{proof}
        \textit{FTSOC,} Suppose $x$ is rational. Then, $x = \frac{a}{b}$ for some integers $a$ and $b$ with no common factors. Raising both sides to the $b$th power, we get $2^a = 3^b$. However, this is a contradiction as $2^a$ is even and $3^b$ is odd. Thus, $x$ is irrational.
    \end{proof}

    \begin{problem}
        Prove that there are no triangular numbers which is two less than a multiple of 11. (Triangular numbers are of the form $n = \frac{k(k+1)}{2}$, where $k$ is a positive integer.)
    \end{problem}

    \begin{proof}
        \textit{FTSOC,} Suppose there exists a triangular number $n = \frac{k(k+1)}{2}$ such that $n \equiv 9 \pmod{11}$. Then, $k(k+1) \equiv 18 \equiv 7 \pmod{11}$. Now, we may consider all the different possible residues of $k$ under mod 11. We have the following table: (see next page)
        \begin{center}
            \begin{tabular}{c|c|c}
                $k \pmod{11}$ & $k(k+1)$ & $k(k+1) \pmod{11}$\\
                \hline
                0 & 0 & 0\\
                1 & 2 & 2\\
                2 & 6 & 6\\
                3 & 12 & 1\\
                4 & 20 & 9\\
                5 & 30 & 8\\
                6 & 42 & 9\\
                7 & 56 & 1\\
                8 & 72 & 6\\
                9 & 90 & 2\\
                10 & 110 & 0\\
            \end{tabular}
        \end{center}
        From the table, we see that there are no possible values of $k$ such that $k(k+1) \equiv 7 \pmod{11}$. Contradiction!
    \end{proof}

    \begin{problem}
        Prove that the sum of a rational number and an irrational number is irrational.
    \end{problem}

    \begin{proof}
        \textit{FTSOC,} Suppose the sum of a rational number $\frac{a}{b}$ and an irrational number $r$ is rational. Then, $\frac{a}{b} + r = q$ for some rational number $q = \frac{c}{d}$. Rearranging, we get \[\frac{a}{b} + r = \frac{c}{d}\] \[r = \frac{c}{d} - \frac{a}{b}\] \[r = \frac{bc - ad}{bd}\] However, since we assumed that $r$ was irrational, it should not have been possible to express it as a fraction. This is a contradiction, so we're done.    
    \end{proof}

    \begin{problem}
        In a party, friendship forms and breaks all the time. However, you discover that no matter what happens, there are always either 3 people who are friends with each other, or 3 people who are not friends with each other. Prove that there are at least 6 people at the party. 
    \end{problem}'



    \begin{proof}
        \textit{FTSOC,} Suppose there are less than $n \leq 5$ people at the party. If we are able to find a construction for 5 people at the party, such that the problem statement is not satisfied, then we would be done as we could choose a subset of size $n$ of the five people to see that this subset also doesn't satisfy the problem. So we go about finding such a construction:
        \begin{center}
            \begin{asy}
                size(3cm);
                draw(unitcircle);
                draw( (0.3,0.3)--(0.3,0.4) );
                draw( (-0.3,0.3)--(-0.3,0.4) );
                draw( (0.6,-0.2)..(0,-0.4)..(-0.6,-0.2) );
            \end{asy}
        \end{center}
    \end{proof}

\end{document}