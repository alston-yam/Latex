%%%%%%%%%%%%%%%%%%%%%%%%%%%%% Define Article %%%%%%%%%%%%%%%%%%%%%%%%%%%%%%%%%%
\documentclass{article}
%%%%%%%%%%%%%%%%%%%%%%%%%%%%%%%%%%%%%%%%%%%%%%%%%%%%%%%%%%%%%%%%%%%%%%%%%%%%%%%

%%%%%%%%%%%%%%%%%%%%%%%%%%%%% Using Packages %%%%%%%%%%%%%%%%%%%%%%%%%%%%%%%%%%
\usepackage{geometry}
\usepackage{graphicx}
\usepackage{amssymb}
\usepackage{amsmath}
\usepackage{amsthm}
\usepackage{empheq}
\usepackage{mdframed}
\usepackage{booktabs}
\usepackage{lipsum}
\usepackage{graphicx}
\usepackage{color}
\usepackage{psfrag}
\usepackage{pgfplots}
\usepackage{bm}
%%%%%%%%%%%%%%%%%%%%%%%%%%%%%%%%%%%%%%%%%%%%%%%%%%%%%%%%%%%%%%%%%%%%%%%%%%%%%%%

% Other Settings

%%%%%%%%%%%%%%%%%%%%%%%%%% Page Setting %%%%%%%%%%%%%%%%%%%%%%%%%%%%%%%%%%%%%%%
\geometry{a4paper}

%%%%%%%%%%%%%%%%%%%%%%%%%% Define some useful colors %%%%%%%%%%%%%%%%%%%%%%%%%%
\definecolor{ocre}{RGB}{243,102,25}
\definecolor{mygray}{RGB}{243,243,244}
\definecolor{deepGreen}{RGB}{26,111,0}
\definecolor{shallowGreen}{RGB}{235,255,255}
\definecolor{deepBlue}{RGB}{61,124,222}
\definecolor{shallowBlue}{RGB}{235,249,255}
%%%%%%%%%%%%%%%%%%%%%%%%%%%%%%%%%%%%%%%%%%%%%%%%%%%%%%%%%%%%%%%%%%%%%%%%%%%%%%%

%%%%%%%%%%%%%%%%%%%%%%%%%% Define an orangebox command %%%%%%%%%%%%%%%%%%%%%%%%
\newcommand\orangebox[1]{\fcolorbox{ocre}{mygray}{\hspace{1em}#1\hspace{1em}}}
%%%%%%%%%%%%%%%%%%%%%%%%%%%%%%%%%%%%%%%%%%%%%%%%%%%%%%%%%%%%%%%%%%%%%%%%%%%%%%%

%%%%%%%%%%%%%%%%%%%%%%%%%%%% English Environments %%%%%%%%%%%%%%%%%%%%%%%%%%%%%
\newtheoremstyle{mytheoremstyle}{3pt}{3pt}{\normalfont}{0cm}{\rmfamily\bfseries}{}{1em}{{\color{black}\thmname{#1}~\thmnumber{#2}}\thmnote{\,--\,#3}}
\newtheoremstyle{myproblemstyle}{3pt}{3pt}{\normalfont}{0cm}{\rmfamily\bfseries}{}{1em}{{\color{black}\thmname{#1}~\thmnumber{#2}}\thmnote{\,--\,#3}}
\theoremstyle{mytheoremstyle}
\newmdtheoremenv[linewidth=1pt,backgroundcolor=shallowGreen,linecolor=deepGreen,leftmargin=0pt,innerleftmargin=20pt,innerrightmargin=20pt,]{theorem}{Theorem}[section]
\theoremstyle{mytheoremstyle}
\newmdtheoremenv[linewidth=1pt,backgroundcolor=shallowBlue,linecolor=deepBlue,leftmargin=0pt,innerleftmargin=20pt,innerrightmargin=20pt,]{definition}{Definition}[section]
\theoremstyle{myproblemstyle}
\newmdtheoremenv[linecolor=black,leftmargin=0pt,innerleftmargin=10pt,innerrightmargin=10pt,]{problem}{Problem}[section]
\theoremstyle{myproblemstyle}
\newmdtheoremenv[linecolor=black,leftmargin=0pt,innerleftmargin=10pt,innerrightmargin=10pt,]{example}{Example}[section]
%%%%%%%%%%%%%%%%%%%%%%%%%%%%%%%%%%%%%%%%%%%%%%%%%%%%%%%%%%%%%%%%%%%%%%%%%%%%%%%

%%%%%%%%%%%%%%%%%%%%%%%%%%%%%%% Plotting Settings %%%%%%%%%%%%%%%%%%%%%%%%%%%%%
\usepgfplotslibrary{colorbrewer}
\pgfplotsset{width=8cm,compat=1.9}
%%%%%%%%%%%%%%%%%%%%%%%%%%%%%%%%%%%%%%%%%%%%%%%%%%%%%%%%%%%%%%%%%%%%%%%%%%%%%%%

%%%%%%%%%%%%%%%%%%%%%%%%%%%%%%% Title & Author %%%%%%%%%%%%%%%%%%%%%%%%%%%%%%%%
\title{Proof by Contradiction}
\author{Alston Yam}
\parindent=0in
%%%%%%%%%%%%%%%%%%%%%%%%%%%%%%%%%%%%%%%%%%%%%%%%%%%%%%%%%%%%%%%%%%%%%%%%%%%%%%%


\begin{document}
    \maketitle

    \section{Introduction}
        Proof by Contradiction is a commonly used tool in Maths Olympiad. The proof starts by assuming the opposite of what you're trying to prove, and by showing that this assumption leads to a contradiction through a series of logical deductions, we can conclude that our initial assumption was false. This then implies that what we were trying to prove is true, so we would be done.
        \bigskip

        Let's use a few examples to see this method in action.
        
        \begin{example}
            Prove that there are infinitely many prime numbers.
        \end{example}

        \textit{Proof.} For the sake of contradiction (FTSOC), assume that there are only finitely many prime numbers. Denote them by a set like so: \[P = \{ p_1, p_2, p_3, \dots, p_n \} \]
        Consider the number $N = p_1 \cdot p_2 \cdot p_3 \cdots p_n + 1$. Since $N$ is not divisible by any of the primes in $P$, it must have another divisor not in $P$. This contradicts our assumption all primes are in $P$. Therefore, there are infinitely many prime numbers. $\blacksquare$

        \begin{example}
            Show that \[ b^2 + b + 1 = a^2 \] has no positive integer solutions.
        \end{example}

        \textit{Proof.} FTSOC, assume that there exists a set of positive integers $(a, b)$ that satisfy the equation. First, we must notice that $b < a$, as otherwise, LHS $>$ RHS (Tiny PbC :0). Then, we can rewrite the equation as: \[ b^2 + b + 1 = a^2 \iff b + 1 = a^2 - b^2 \iff b + 1 = (a-b)(a+b) \]
        We know that $a > b \geq 1$. This means that $a - b \geq 1$, and $a + b \geq b + 2$, which when combined gives $(a-b)(a+b) \geq b+2$. But we have $b + 1 = (a-b)(a+b) \geq b+2$, contradiction! $\blacksquare$
        
        \begin{example}
            Prove that math is the best subject.
        \end{example}
        \textit{Proof.} FTSOC, assume that math is not the best subject. This implies that there exists a subject that is better than math, absurd! $\blacksquare$

    \clearpage

    \section{When to use Proof by Contradiction?}
        Hopefully from the previous section, you now have a general idea of how a Proof by Contradiction works, and how to write up one in your solution. Here, I will go over some common patterns that emerge in problems where a Proof by Contradiction is appropriate: 
        \begin{itemize}
            \item When the statement ask you to prove that there exist a P that satisfy some property Q.
            \item When the problem statement ask you to prove the existence of an infinite number of something (refer back to Example 1.1).
            \item When the statement gives you very little information about what you're trying to prove.
        \end{itemize}

        In particular, I find the first point really useful. This is because we can now work with the condition that ``there are no number P that satisfy property Q'', which is quite cool.
    \section{Problems}
    \bigskip

    \begin{problem}
        Prove that $\sqrt{2}$ is irrational.
    \end{problem}

    \begin{problem}
        Prove that $\sqrt[m]{n}$ is either irrational or an integer $\forall n, m \in \mathbb{N}$.
    \end{problem}

    \begin{problem}
        Show that if $2^x = 3$, then $x$ is irrational.
    \end{problem}

    \begin{problem}
        Prove that there are no triangular numbers which is one less than a multiple of 11. (Triangular numbers are of the form $n = \frac{k(k+1)}{2}$, where $k$ is a positive integer.)
    \end{problem}

    \begin{problem}
        Prove that the sum of a rational number and an irrational number is irrational.
    \end{problem}

    \begin{problem}
        In a party, friendship forms and breaks all the time. However, you discover that no matter what happens, there are always either 3 people who are friends with each other, or 3 people who are not friends with each other. Prove that there are at least 6 people at the party.
    \end{problem}

\end{document}

