%%%%%%%%%%%%%%%%%%%%%%%%%%%%% Define Article %%%%%%%%%%%%%%%%%%%%%%%%%%%%%%%%%%
\documentclass{article}
%%%%%%%%%%%%%%%%%%%%%%%%%%%%%%%%%%%%%%%%%%%%%%%%%%%%%%%%%%%%%%%%%%%%%%%%%%%%%%%

%%%%%%%%%%%%%%%%%%%%%%%%%%%%% Using Packages %%%%%%%%%%%%%%%%%%%%%%%%%%%%%%%%%%
\usepackage{geometry}
\usepackage{graphicx}
\usepackage{amssymb}
\usepackage{amsmath}
\usepackage{amsthm}
\usepackage{empheq}
\usepackage{mdframed}
\usepackage{booktabs}
\usepackage{lipsum}
\usepackage{graphicx}
\usepackage{color}
\usepackage{psfrag}
\usepackage{pgfplots}
\usepackage{bm}
\usepackage{dirtytalk}
%%%%%%%%%%%%%%%%%%%%%%%%%%%%%%%%%%%%%%%%%%%%%%%%%%%%%%%%%%%%%%%%%%%%%%%%%%%%%%%

% Other Settings

%%%%%%%%%%%%%%%%%%%%%%%%%% Page Setting %%%%%%%%%%%%%%%%%%%%%%%%%%%%%%%%%%%%%%%
\geometry{a4paper}

%%%%%%%%%%%%%%%%%%%%%%%%%% Define some useful colors %%%%%%%%%%%%%%%%%%%%%%%%%%
\definecolor{ocre}{RGB}{243,102,25}
\definecolor{mygray}{RGB}{243,243,244}
\definecolor{deepGreen}{RGB}{26,111,0}
\definecolor{shallowGreen}{RGB}{235,255,255}
\definecolor{deepBlue}{RGB}{61,124,222}
\definecolor{shallowBlue}{RGB}{235,249,255}
%%%%%%%%%%%%%%%%%%%%%%%%%%%%%%%%%%%%%%%%%%%%%%%%%%%%%%%%%%%%%%%%%%%%%%%%%%%%%%%

%%%%%%%%%%%%%%%%%%%%%%%%%% Define an orangebox command %%%%%%%%%%%%%%%%%%%%%%%%
\newcommand\orangebox[1]{\fcolorbox{ocre}{mygray}{\hspace{1em}#1\hspace{1em}}}
%%%%%%%%%%%%%%%%%%%%%%%%%%%%%%%%%%%%%%%%%%%%%%%%%%%%%%%%%%%%%%%%%%%%%%%%%%%%%%%

%%%%%%%%%%%%%%%%%%%%%%%%%%%% English Environments %%%%%%%%%%%%%%%%%%%%%%%%%%%%%
\newtheoremstyle{mytheoremstyle}{3pt}{3pt}{\normalfont}{0cm}{\rmfamily\bfseries}{}{1em}{{\color{black}\thmname{#1}~\thmnumber{#2}}\thmnote{\,--\,#3}}
\newtheoremstyle{myproblemstyle}{3pt}{3pt}{\normalfont}{0cm}{\rmfamily\bfseries}{}{1em}{{\color{black}\thmname{#1}~\thmnumber{#2}}\thmnote{\,--\,#3}}
\theoremstyle{mytheoremstyle}
\newmdtheoremenv[linewidth=1pt,backgroundcolor=shallowGreen,linecolor=deepGreen,leftmargin=0pt,innerleftmargin=20pt,innerrightmargin=20pt,]{theorem}{Theorem}[section]
\theoremstyle{mytheoremstyle}
\newmdtheoremenv[linewidth=1pt,backgroundcolor=shallowBlue,linecolor=deepBlue,leftmargin=0pt,innerleftmargin=20pt,innerrightmargin=20pt,]{definition}{Definition}[section]
\theoremstyle{myproblemstyle}
\newmdtheoremenv[linecolor=black,leftmargin=0pt,innerleftmargin=10pt,innerrightmargin=10pt,]{problem}{Problem}[section]
\theoremstyle{myproblemstyle}
\newmdtheoremenv[linecolor=black,leftmargin=0pt,innerleftmargin=10pt,innerrightmargin=10pt,]{example}{Example}[section]
%%%%%%%%%%%%%%%%%%%%%%%%%%%%%%%%%%%%%%%%%%%%%%%%%%%%%%%%%%%%%%%%%%%%%%%%%%%%%%%

%%%%%%%%%%%%%%%%%%%%%%%%%%%%%%% Plotting Settings %%%%%%%%%%%%%%%%%%%%%%%%%%%%%
\usepgfplotslibrary{colorbrewer}
\pgfplotsset{width=8cm,compat=1.9}
%%%%%%%%%%%%%%%%%%%%%%%%%%%%%%%%%%%%%%%%%%%%%%%%%%%%%%%%%%%%%%%%%%%%%%%%%%%%%%%

%%%%%%%%%%%%%%%%%%%%%%%%%%%%%%% Title & Author %%%%%%%%%%%%%%%%%%%%%%%%%%%%%%%%
\title{Lessons in Maths Olympiads}
\author{Alston Yam}
\parskip=5pt
%%%%%%%%%%%%%%%%%%%%%%%%%%%%%%%%%%%%%%%%%%%%%%%%%%%%%%%%%%%%%%%%%%%%%%%%%%%%%%%

\begin{document}
    \maketitle
    \parindent=0pt
    \section{Introduction}
    This document aims to summarise all of the lessons that I've learnt and the mistakes I've made during my preparation for the 2025 IMO.

    \section{Lessons}
    \subsection{Hands Dirty}

    \subsection{Wishful Thinking}
    \vspace{3pt}
    \begin{center}
        \say{Ohhhh... wouldn't it be SOOOOOO nice if this was true?}
    \end{center}

    Often times it is beneficial to employ this technique to look \say{into the future}, to try and find conjectures that we wish to proof that would further our investigation.

    This comes in many forms: Hoping that four points are cyclic, this sequence is bounded, the function is injective $\dots$

    Let's look at an example.

    \begin{example}[2021 ISL N1]
        Find all positive integers $n\geq1$ such that there exists a pair $(a,b)$ of positive integers, such that $a^2+b+3$ is not divisible by the cube of any prime, and \[ n=\frac{ab+3b+8}{a^2+b+3}. \]
    \end{example}

    My first thought was to write $a^2 + b + 3 \mid ab + 3b + 8$, and start cancelling terms. This was certainly the right idea, however without the following motivation, I would be running into deadends left right and center not knowing what I should cancel.
    
    \begin{center}
        \say{Huh.. the degree of $b$ on both parts of the fraction is 1, so wouldn't it be nice if I had a fraction in just $a$?}
    \end{center}

    With this motivation we do the following:

    \begin{align*}
        &a^2 + b + 3 \mid b(a+3) + 8 \\
        &a^2 + b + 3 \mid b(a+3) + 8 - (a+3)(a^2 + b + 3)\\
        &a^2 + b + 3 \mid b(a+3) + 8 - a^2(a+3) - b(a+3) - 3(a+3)\\
        &a^2 + b + 3 \mid -(a+1)^3
    \end{align*}
    Which is actually really nice, as we know $a^2 + b + 3$ is cube free. So hence we have $a^2 + b + 3 \mid (a+1)^2$. But $a^2 + 2b + 5 > 2a$ for any positive integers $a$, which gives $2(a^2 + b + 3) > (a+1)^2$. therefore we must have $(a+1)^2 = a^2 + b + 3 \iff a = 2b - 2$. We can now sub back into the original equation to get $n=2$ is the only solution, and we're done.

    


    \subsection{Visualisation}
    % that one A1 that asks for the unique value of a_n that satisfies this equation...
    This will come pretty naturally in most combinatorics questions, however we are not restricted to just that. As a learn-by-picture guy myself I often times find turning things such as sequences into something that I can see in my head helps me solve problems much faster.


    
    
\end{document}