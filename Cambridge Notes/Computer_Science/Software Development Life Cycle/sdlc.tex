%%%%%%%%%%%%%%%%%%%%%%%%%%%%% Define Article %%%%%%%%%%%%%%%%%%%%%%%%%%%%%%%%%%
\documentclass{article}
%%%%%%%%%%%%%%%%%%%%%%%%%%%%%%%%%%%%%%%%%%%%%%%%%%%%%%%%%%%%%%%%%%%%%%%%%%%%%%%

%%%%%%%%%%%%%%%%%%%%%%%%%%%%% Using Packages %%%%%%%%%%%%%%%%%%%%%%%%%%%%%%%%%%
\usepackage{geometry}
\usepackage{graphicx}
\usepackage{amssymb}
\usepackage{amsmath}
\usepackage{amsthm}
\usepackage{empheq}
\usepackage{mdframed}
\usepackage{booktabs}
\usepackage{lipsum}
\usepackage{graphicx}
\usepackage{psfrag}
\usepackage{pgfplots}
\usepackage{bm}
\usepackage{color}
%%%%%%%%%%%%%%%%%%%%%%%%%%%%%%%%%%%%%%%%%%%%%%%%%%%%%%%%%%%%%%%%%%%%%%%%%%%%%%%

% Other Settings

%%%%%%%%%%%%%%%%%%%%%%%%%% Page Setting %%%%%%%%%%%%%%%%%%%%%%%%%%%%%%%%%%%%%%%
\geometry{a4paper}

%%%%%%%%%%%%%%%%%%%%%%%%%% Define some useful colors %%%%%%%%%%%%%%%%%%%%%%%%%%
\definecolor{ocre}{RGB}{243,102,25}
\definecolor{mygray}{RGB}{243,243,244}
\definecolor{deepGreen}{RGB}{26,111,0}
\definecolor{shallowGreen}{RGB}{235,255,255}
\definecolor{deepBlue}{RGB}{61,124,222}
\definecolor{shallowBlue}{RGB}{235,249,255}
%%%%%%%%%%%%%%%%%%%%%%%%%%%%%%%%%%%%%%%%%%%%%%%%%%%%%%%%%%%%%%%%%%%%%%%%%%%%%%%

%%%%%%%%%%%%%%%%%%%%%%%%%% Define an orangebox command %%%%%%%%%%%%%%%%%%%%%%%%
\newcommand\orangebox[1]{\fcolorbox{ocre}{mygray}{\hspace{1em}#1\hspace{1em}}}
%%%%%%%%%%%%%%%%%%%%%%%%%%%%%%%%%%%%%%%%%%%%%%%%%%%%%%%%%%%%%%%%%%%%%%%%%%%%%%%

%%%%%%%%%%%%%%%%%%%%%%%%%%%% English Environments %%%%%%%%%%%%%%%%%%%%%%%%%%%%%
\newtheoremstyle{mytheoremstyle}{3pt}{3pt}{\normalfont}{0cm}{\rmfamily\bfseries}{}{1em}{{\color{black}\thmname{#1}~\thmnumber{#2}}\thmnote{\,--\,#3}}
\newtheoremstyle{myproblemstyle}{3pt}{3pt}{\normalfont}{0cm}{\rmfamily\bfseries}{}{1em}{{\color{black}\thmname{#1}~\thmnumber{#2}}\thmnote{\,--\,#3}}
\theoremstyle{mytheoremstyle}
\newmdtheoremenv[linewidth=1pt,backgroundcolor=shallowGreen,linecolor=deepGreen,leftmargin=0pt,innerleftmargin=20pt,innerrightmargin=20pt,]{theorem}{Theorem}[section]
\theoremstyle{mytheoremstyle}
\newmdtheoremenv[linewidth=1pt,backgroundcolor=shallowBlue,linecolor=deepBlue,leftmargin=0pt,innerleftmargin=20pt,innerrightmargin=20pt,]{definition}{Definition}[section]
\theoremstyle{myproblemstyle}
\newmdtheoremenv[linecolor=black,leftmargin=0pt,innerleftmargin=10pt,innerrightmargin=10pt,]{problem}{Problem}[section]
%%%%%%%%%%%%%%%%%%%%%%%%%%%%%%%%%%%%%%%%%%%%%%%%%%%%%%%%%%%%%%%%%%%%%%%%%%%%%%%

%%%%%%%%%%%%%%%%%%%%%%%%%%%%%%% Plotting Settings %%%%%%%%%%%%%%%%%%%%%%%%%%%%%
\usepgfplotslibrary{colorbrewer}
\pgfplotsset{width=8cm,compat=1.9}
%%%%%%%%%%%%%%%%%%%%%%%%%%%%%%%%%%%%%%%%%%%%%%%%%%%%%%%%%%%%%%%%%%%%%%%%%%%%%%%

%%%%%%%%%%%%%%%%%%%%%%%%%%%%%%% Title & Author %%%%%%%%%%%%%%%%%%%%%%%%%%%%%%%%
\title{Software Development Life Cycle (SDLC)}
\author{Alston}
\date{}

\parindent=0pt
\parskip=3pt
%%%%%%%%%%%%%%%%%%%%%%%%%%%%%%%%%%%%%%%%%%%%%%%%%%%%%%%%%%%%%%%%%%%%%%%%%%%%%%%

\begin{document}
    \maketitle

    \section{Introduction}
    Notes on SDLC for the 9618 CAIE Computer Science course.

    \section{Basics}
    The five most crucial steps are: $\textbf{Analysis, Design, Implementation, Evaluation, Maintenance}$.

    \subsection{Analysis}
    Define the problem. Do the research and collect data. 

    \subsection{Design}
    Outline the program process: structure charts, what each module does, what parameters are being passed. Decide what hardware is required. Also outline the UI and UX

    \subsection{Implementation/programming}
    Software developer writes and debugs each module

    \subsection{Implementation/testing}

    \begin{center}
        \begin{tabular}{|l|p{8cm}|}
        \hline
        \textbf{Testing Type} & \textbf{Description} \\
        \hline
        White-box Testing & Tests the internal structure and logic of the program; requires knowledge of the source code. \\
        \hline
        Black-box Testing & Tests the program’s functionality without knowing the internal code; focuses on inputs and expected outputs. \\
        \hline
        Unit Testing & Tests individual modules or components of the system in isolation. \\
        \hline
        Integration Testing & Tests how different modules work together and communicate correctly. \\
        \hline
        Alpha Testing & Conducted by the developers or in-house testers before release, to identify and fix major issues. \\
        \hline
        Beta Testing & Conducted by a limited group of end users in a real environment before final release. \\
        \hline
        Acceptance Testing & Determines whether the system meets the agreed requirements and is ready for deployment; often done by the client. \\
        \hline
        \end{tabular}
    \end{center}

    \subsection{Evaluation}
    Normally carried out 3-6 months after the software is online. End user feedback is collected about UI and UX and functionality.

    \subsection{Maintenance}
    Software upgrades and bug fixes. Corrective, adaptive, perfective maintenance. $\textbf{Corrective}$: fixes bugs that were not found in previous stages of testing. $\textbf{Adaptive}$: development of changes that were necessary for the system. Program is changed to do something it's not orginally designed to do. $\textbf{Perfective}$: Quality of life improvements


    \section{Waterfall Life Cycle Model}
    Each step follows from each other linearly. Not a flexible model. It's good for the manufacturing industry, cuz changes after the implementation stage will icnrease the cost, so we must take great care before moving on.

    \section{Spiral and Agile}
    Spiral model just runs the fives stage process over and over again for many times. After each run, a prototype might be generated and improved upon. Best for long term projects.

    Agile model is non-linear, so the developer makes some changes and gets feedback from the users. Best for short term projects.
    
\end{document}