%%%%%%%%%%%%%%%%%%%%%%%%%%%%% Define Article %%%%%%%%%%%%%%%%%%%%%%%%%%%%%%%%%%
\documentclass{article}
%%%%%%%%%%%%%%%%%%%%%%%%%%%%%%%%%%%%%%%%%%%%%%%%%%%%%%%%%%%%%%%%%%%%%%%%%%%%%%%

%%%%%%%%%%%%%%%%%%%%%%%%%%%%% Using Packages %%%%%%%%%%%%%%%%%%%%%%%%%%%%%%%%%%
\usepackage{geometry}
\usepackage{graphicx}
\usepackage{amssymb}
\usepackage{amsmath}
\usepackage{amsthm}
\usepackage{empheq}
\usepackage{mdframed}
\usepackage{booktabs}
\usepackage{lipsum}
\usepackage{graphicx}
\usepackage{color}
\usepackage{psfrag}
\usepackage{pgfplots}
\usepackage{bm}
%%%%%%%%%%%%%%%%%%%%%%%%%%%%%%%%%%%%%%%%%%%%%%%%%%%%%%%%%%%%%%%%%%%%%%%%%%%%%%%

% Other Settings

%%%%%%%%%%%%%%%%%%%%%%%%%% Page Setting %%%%%%%%%%%%%%%%%%%%%%%%%%%%%%%%%%%%%%%
\geometry{a4paper}

%%%%%%%%%%%%%%%%%%%%%%%%%% Define some useful colors %%%%%%%%%%%%%%%%%%%%%%%%%%
\definecolor{ocre}{RGB}{243,102,25}
\definecolor{mygray}{RGB}{243,243,244}
\definecolor{deepGreen}{RGB}{26,111,0}
\definecolor{shallowGreen}{RGB}{235,255,255}
\definecolor{deepBlue}{RGB}{61,124,222}
\definecolor{shallowBlue}{RGB}{235,249,255}
%%%%%%%%%%%%%%%%%%%%%%%%%%%%%%%%%%%%%%%%%%%%%%%%%%%%%%%%%%%%%%%%%%%%%%%%%%%%%%%

%%%%%%%%%%%%%%%%%%%%%%%%%% Define an orangebox command %%%%%%%%%%%%%%%%%%%%%%%%
\newcommand\orangebox[1]{\fcolorbox{ocre}{mygray}{\hspace{1em}#1\hspace{1em}}}
\newcommand{\f}[2]{\frac{#1}{#2}}
\DeclareMathOperator{\sech}{sech}
\DeclareMathOperator{\csch}{csch}
\newcommand{\ddx}{\frac{d}{dx}}
\newcommand{\dd}[2]{\frac{d #1}{d #2}}
\renewcommand{\arcsin}{\sin^{-1}}
\renewcommand{\arccos}{\cos^{-1}}
\renewcommand{\arctan}{\tan^{-1}}
\newcommand{\artanh}{\tanh^{-1}}
\newcommand{\arsinh}{\sinh^{-1}}
\newcommand{\arcosh}{\cosh^{-1}}
\newcommand{\intg}[2]{\int #1 \, d#2}
%%%%%%%%%%%%%%%%%%%%%%%%%%%%%%%%%%%%%%%%%%%%%%%%%%%%%%%%%%%%%%%%%%%%%%%%%%%%%%%

%%%%%%%%%%%%%%%%%%%%%%%%%%%% English Environments %%%%%%%%%%%%%%%%%%%%%%%%%%%%%
\newtheoremstyle{mytheoremstyle}{3pt}{3pt}{\normalfont}{0cm}{\rmfamily\bfseries}{}{1em}{{\color{black}\thmname{#1}~\thmnumber{#2}}\thmnote{\,--\,#3}}
\newtheoremstyle{myproblemstyle}{3pt}{3pt}{\normalfont}{0cm}{\rmfamily\bfseries}{}{1em}{{\color{black}\thmname{#1}~\thmnumber{#2}}\thmnote{\,--\,#3}}
\theoremstyle{mytheoremstyle}
\newmdtheoremenv[linewidth=1pt,backgroundcolor=shallowGreen,linecolor=deepGreen,leftmargin=0pt,innerleftmargin=20pt,innerrightmargin=20pt,]{theorem}{Theorem}[section]
\theoremstyle{mytheoremstyle}
\newmdtheoremenv[linewidth=1pt,backgroundcolor=shallowBlue,linecolor=deepBlue,leftmargin=0pt,innerleftmargin=20pt,innerrightmargin=20pt,]{definition}{Definition}[section]
\theoremstyle{myproblemstyle}
\newmdtheoremenv[linecolor=black,leftmargin=0pt,innerleftmargin=10pt,innerrightmargin=10pt,]{problem}{Problem}[section]
\theoremstyle{myproblemstyle}
\newmdtheoremenv[linecolor=black,leftmargin=0pt,innerleftmargin=10pt,innerrightmargin=10pt,]{example}{Example}[section]
%%%%%%%%%%%%%%%%%%%%%%%%%%%%%%%%%%%%%%%%%%%%%%%%%%%%%%%%%%%%%%%%%%%%%%%%%%%%%%%

%%%%%%%%%%%%%%%%%%%%%%%%%%%%%%% Plotting Settings %%%%%%%%%%%%%%%%%%%%%%%%%%%%%
\usepgfplotslibrary{colorbrewer}
\pgfplotsset{width=8cm,compat=1.9}
\parindent=0pt
\parskip=3pt
%%%%%%%%%%%%%%%%%%%%%%%%%%%%%%%%%%%%%%%%%%%%%%%%%%%%%%%%%%%%%%%%%%%%%%%%%%%%%%%

%%%%%%%%%%%%%%%%%%%%%%%%%%%%%%% Title & Author %%%%%%%%%%%%%%%%%%%%%%%%%%%%%%%%
\title{9231 Further Maths Pure — Calculus}
\author{Alston}
\date{}
%%%%%%%%%%%%%%%%%%%%%%%%%%%%%%%%%%%%%%%%%%%%%%%%%%%%%%%%%%%%%%%%%%%%%%%%%%%%%%%

\begin{document}
    \maketitle
    \section{Introduction}
    My compiled notes of the calculus section for the CAIE Further Maths Course.

    \section{Differentiation}
    \subsection{Implicit Differentiation}
    Sometimes it's difficult to obtain $f'(x)$ directly from the expression, so you would differentiate both sides w.r.t. to $x$, where $\frac{d}{dx}(y) = \frac{dy}{dx}$. Other expressions including $y$ are just chain rule.

    \begin{example}
        Find first and second derivative of $e^x + e^{2y} = \ln{y}$ w.r.t. $x$.
    \end{example}

    \begin{align*}
    \f{d}{dx}(e^x + e^{2y}) &= \f{d}{dx}(\ln{y})\\
    e^x + 2e^{2y}\f{dy}{dx} &= \f{1}{y} \frac{dy}{dx}\\
    \frac{dy}{dx} &= \f{ye^x}{1-2ye^{2y}}
    \end{align*}
    

    The second derivative follows as well with the same method (use product rule).
    \begin{theorem}[Triple Product Rule]
        If $y = uvw$, then $\frac{dy}{dx} = uv\frac{dw}{dx} + uw\frac{dv}{dx} + vw\frac{du}{dx}$
    \end{theorem}

    Proof is by first principles.

    \begin{example}
        An implicit curve is defined as $\sin{2x}\cos{2y} = \f{\sqrt{3}}{4}$. It is known that the curve passes through the point $P(\f{\pi}{3}, \f{\pi}{6})$. Find, at point P, $\f{dy}{dx}$ and $\f{d^2y}{dx^2}$.
    \end{example}


    \[2\cos{2x}\cos{2y} + \sin{2x}\cdot(-2\f{dy}{dx}\sin{2y}) = 0\]
    \[\f{dy}{dx} = \cot{2x}\cot{2y}\]

    \[\f{d^2y}{dx^2} = -2\csc{2x}\cdot\cot{2y} + \cot{2x}\cdot (-2\csc{2y})\cdot\f{dy}{dx}\]
    then we can substitute the values for $(x, y)$ and $\f{dy}{dx}$ here.
  
    \subsection{Parametric Differentiation}
    \begin{theorem}[Formulae]
        For parametric equations $x = f(t)$, $y = g(t)$, we have
        \[\f{dy}{dx} = \f{d}{dt}(y) \times \f{dt}{dx}\]
        \[\f{d^2y}{dx^2} = \f{d}{dt}(\f{dy}{dx}) \times \f{dt}{dx}\]
    \end{theorem}
    Here, to figure out the second derivative, we have essentially just done the exact same `algorithm' on the first derivative. So the method generalises to any higher order derivatives.

    \begin{example}
        Consider $x = \cos{t}, y = \sin{2t}$ for $0\leq t \leq \pi$. Find coordinates of all stationary points and determine their nature.
    \end{example}

    We know $\f{dx}{dt} = -\sin{t}$, $\f{dy}{dt} = 2\cos{2t}$. So $\f{dy}{dx} = \f{-2\cos{2t}}{\sin{t}}$. Set $\f{dy}{dx} = 0$ then we get that $t = \f{\pi}{4}, \f{3\pi}{4}$. Substitute each of these $t$ values back into the original equation to get the coordinates in $x$ and $y$.

    $\f{d}{dt}(\frac{dy}{dx}) = -\f{\sin{t}(-4 \sin{2t}) - 2\cos{2t}(\cos{t})}{\sin^2{t}}$, then we just multiply by $\f{dt}{dx} = -\f{1}{\sin{t}}$ to get the second derivative. Sub in values of $t$, and if second derivative is $>$ than $0$ then the point is a minimum. 

    Note that if second derivative is $0$, then the point $\textit{could}$ be an inflection point, we have to look at the third derivative, and if third derivative is not $0$, then the point is an inflection, otherwise it's not.

    \subsection{Hyperbolic and Inverse Functions}
    \begin{definition}[Hyperbolic Functions]
        \[\sinh{x} = \frac{e^x - e^{-x}}{2}\]
        \[\cosh{x} = \frac{e^x + e^{-x}}{2}\]
        \[\tanh{x} = \frac{e^x - e^{-x}}{e^x + e^{-x}}\]
    \end{definition}

    \begin{theorem}[Elementary Derivatives]
        \[\f{d}{dx}(\sinh{x}) = \cosh{x}, \f{d}{dx}(\cosh{x}) = \sinh{x}\]
    \end{theorem}

    These are kinda similar to trig, but not exactly the same.

    Recall the identity $\cosh^2{x} - \sinh^2{x} = 1$. 

    \vspace{10pt}
    \begin{theorem}[Derivatives of other hyperbolic functions]
        \[\f{d}{dx}(\tanh{x}) = \f{\cosh{x}\cosh{x} - \sinh{x}\sinh{x}}{\cosh^2{x}} = \sech^2{x}\]
        \[\ddx (\sech{x}) = -\sech{x}\tanh{x}\]
        \[\ddx (\csch{x}) = -\csch{x}\coth{x}\]
        \[\ddx (\coth{x}) = -\csch^2{x}\]
    \end{theorem}

    We can also differentiate inverses of trig and hyperbolic functions with implicit differentiation.


    \[
    \begin{array}{c|c|c}
    \text{Function} & \text{Derivative} & \text{Logarithmic Form} \\ \hline

    \sin^{-1} x & \dfrac{1}{\sqrt{1 - x^2}} & \\[1.2em]

    \cos^{-1} x & -\dfrac{1}{\sqrt{1 - x^2}} & \\[1.2em]

    \tan^{-1} x & \dfrac{1}{1 + x^2} & \\[1.2em]

    \arsinh{x} & \dfrac{1}{\sqrt{x^2 + 1}} 
    & \arsinh{x} = \ln\!\left(x + \sqrt{x^2 + 1}\right) \\[1.2em]

    \arcosh{x} & \dfrac{1}{\sqrt{x^2 - 1}} 
    & \arcosh{x} = \ln\!\left(x + \sqrt{x^2 - 1}\right) \quad (x > 1) \\[1.2em]

    \artanh{x} & \dfrac{1}{1 - x^2} 
    & \artanh{x} = \dfrac12 \ln\!\left(\dfrac{1 + x}{1 - x}\right) \quad (|x| < 1)
    \end{array}
    \]

    This table will be useful for u-subs in integration.



    \subsection{Taylor \& McLaurin Series}
    Taylor Series approximates a function with a polynomial. If the polynomial goes up to infinite degree, it's called a series, otherwise it's called a Taylor Polynomial.

    \begin{theorem}[Taylor Series]
        Given a function $f(x)$, we can approximate it at any point $a$ as
        \[f(x) = f(a) + \dd{f}{x}(a)\f{(x-a)}{1!} + \frac{d^2f}{dx^2}(a)\f{(x-a)^2}{2!} + \frac{d^3f}{dx^3}(a)\f{(x-a)^3}{3!} + \cdots\]
    \end{theorem}

    So essentially, I am just making sure that for any arbitrary point $a$, the $0$th, $1$st, $2$nd etc derivatives of the approximiating polynomial matches up exactly with that of the function $f(x)$. The McLaurin Series is a special case of the Taylor Series when we have $a = 0$. 

    Notable approximations using a McLaurin Series are $e^x, \sin{x}, \cos{x}$. 

    
    \section{Integration}
    We start off with a really useful result:
    \begin{theorem}[Integration by Parts]
        \[\intg{uv'}{x} = uv + \intg{u'v}{x}\] 
    \end{theorem}
    This is the analogue to `product rule'. 

    To choose the appropriate $u$, use priority based on `I LATE': inverse trig functions ($\arcsin{x}$), logs($\ln{x}$), arithmetics($x^2$), trig($\sin{x}$), exponentials($e^x$).
    
    \subsection{Integration Techniques}
    When we integrate fractions with square roots on the bottom, they look suspiciously similar to the derivatives of the trig functions. So we can actually use trig sub here.

    If there's a $\textbf{square root}$ on the bottom, use $\sin, \cos, \sinh, \cosh$, otherwise use $\tan, \tanh$. 

    Inside the squareroot in the bottom, if the $\textbf{coefficient of $x^2$}$ is negative, we use normal trig ($\sin, \cos$), otherwise we use hyperbolics ($\sinh, \cosh$). 

    \begin{example}
        Use an appropriate substitution, determine the result of $I = \intg{\f{3}{\sqrt{x^2 + 4}}}{x}$. 
    \end{example}
    
    Consider the sub $x = 2\sinh{u}$. $dx = 2\cosh{(u)}du$. Hence we have \[I = \intg{\frac{3}{\sqrt{x^2 + 4}}}{x} = \int\frac{3}{\sqrt{4\sinh^2{u} + 4}} \cdot 2\cosh{(u)} du = \intg{3}{u} = 3u + C\]

    We can also use complete the square.

    \begin{example}
        Use a substitution to determine $I = \intg{\frac{1}{x^2 - 4x + 7}}{x}$
    \end{example}

    Notice that $x^2 - 4x + 7 = (x-2)^2 + 3$. So we let $\sqrt{3}\tan{u} = (x-2) \implies dx = \sqrt{3}\sec^2{u}$,  which will give us 
    \[I = \intg{\frac{1}{(x-2)^2 + 3}}{x} = \intg{\frac{1}{3(\tan^2{u} + 1)}\cdot \sqrt{3}\sec^2{u}}{u} = \intg{\frac{1}{\sqrt{3}}}{u} = \f{1}{\sqrt{3}}u + C\]

    To integrate inverse hyperbolic functions, we will need to use IBP.

    \begin{example}
        Find \[I_n = \intg{\arsinh{x}}{x}\]
    \end{example}

    To solve this, we would set $u = \arsinh{x}$, $v' = 1$ and use IBP.


    \subsection{Integration by Reduction}
    \begin{example}
        Find \[I = \intg{\sin^n{x}}{x}\]
    \end{example}

    We let $u = \sin^{n-1}{x}$ and $v' = \sin{x}$. Then we can use IBP:
    
    \begin{align*}
        I_n &= -\cos{x}\sin^{n-1}{x} - \intg{(n-1)\cos{x}\sin^{n-2}{x}\cdot (-\cos{x})}{x}\\
        &= (n-1)\intg{\cos^2{x} \cdot \sin^{n-2}{x}}{x} - \cos{x}\sin^{n-1}{x}\\
        &= (n-1)\intg{(1-\sin^2{x})\sin^{n-2}{x}}{x} - \cos{x}\sin^{n-1}{x}\\
        &= (n-1)I_{n-2} - (n-1)I_n - \cos{x}\sin^{n-1}{x}
    \end{align*}
    
    \begin{example}
        Find \[I_n = \intg{\tanh^n{x}}{x}\]
    \end{example}

    Note that $\tanh^n{x} = \tanh^{n-2}{x}(1 - \sech^2{x}) = \tanh^{n-2}{x} - \sech^2{x}\tanh^{n-2}{x}$. Hence \[I_n = I_{n-2} - \intg{sech^2{x}\tanh^{n-2}{x}}{x}\]

    Now we let $u = \tanh^{n-2}{x}$ and $v' = \sech^2{x}$, and use IBP.

    \begin{align*}
        \intg{sech^2{x}\tanh^{n-2}{x}}{x} &= \tanh^{n-1}{x} - \intg{(n-2)\cdot \tanh^{n-2}{x}\sech^2{x}}{x}\\
        (n-1) \intg{sech^2{x}\tanh^{n-2}{x}}{x} &= \tanh^{n-1}{x}\\
        \intg{sech^2{x}\tanh^{n-2}{x}}{x} &= \f{1}{n-1}\tanh^{n-1}{x}
    \end{align*}

    Substitute back we find \[I_n = I_{n-2} - \f{1}{n-1}\tanh^{n-1}{x}\]

    \begin{theorem}[Important Note]
        When using the reduction formulae, if we are trying to integrate $\sin, \cos, \sinh, \cosh$, then we go down by $1$, otherwise for $\tan, \tanh$ we go down by $2$.
    \end{theorem}

    
    
    \subsection{Arc Length and Surface Areas}
        \begin{definition}[Arc Lengths]
            \[s = \int_{a}^{b} \sqrt{1 + \left(\dd{y}{x}\right)^2} \, dx\]
            \[s = \int_{a}^{b} \sqrt{{\left(\dd{x}{t}\right)^2} + \left(\dd{y}{t}\right)^2} \, dt\]
            \[s = \int_{\theta_1}^{\theta_2} \sqrt{r^2 + \left(\dd{r}{\theta}\right)^2} \, d\theta\]
        \end{definition}

        To find the equivalent formulae for surface area, multiply by $2\pi$ and then also by the other axis that you're rotating through. e.g: rotating through the $x$ axis, the formula would be $s = \int_{a}^{b} 2\pi y \sqrt{1 + \left(\dd{y}{x}\right)^2} \, dx$


    \subsection{Limits of Areas}
    To find bounds of the expression $\displaystyle\sum_{n=1}^{\infty}f(n)$, we can think about it as rectanges having height $f(n)$, and then we can fit a function onto it and then integrate. This test is called the $\textbf{integral test}$ and it can be used to test if a series will converge or diverge.

    \begin{example}
        Determine if $\displaystyle\sum_{n=1}^{\infty}\f{1}{e^n}$ converges or diverges.
    \end{example}

    We consider an arbitrarily large $N$. We know that \[\displaystyle\sum_{n=1}^{N} \frac{1}{e^n} < \int_0^{N}\f{1}{e^x} dx = 1-\frac{1}{e^N}\]
    So as $N$ approaches infinity, we actually have the integral approaching to $1$, which means the summation will converge. 

    We can also use the comparison test: for wack functions like $\frac{1}{x^2 \ln{x}}$, we can just compare it to $\frac{1}{x^2}$ or $\frac{1}{x^3}$.

    
    
    \section{Differential Equations}
    \subsection{First Order}
    Consider a general differential equation of the form $\dd{y}{x} + Fy = G$ where $F, G$ are functions in $x$. The idea is we want to compare this differential equation with product rule. Multiply through by $I(x)$

    \[I\dd{y}{x} + FIy = GI\]

    Compare that with $vdu + udv$, if $v = y$, then we would like to have $u = I$. This means that $I$ must be a function such that $\dd{I}{x} = FI$. 

    \[\intg{\f{1}{I}}{I} = \intg{F}{x} \iff I = e^{\intg{F}{x}}\]

    Here, $I$ is the $\textbf{integrating factor}$.

    \subsection{Second Order: Homogeneous}
    We need two concepts: $\textbf{auxiliary equation}$ and $\textbf{complementary function}$. 

    \begin{example}
        Find the complementary function to $\frac{d^2y}{dx^2} + 3\frac{dy}{dx} - 10y = 0$.
    \end{example}

    We first consider our trial solution $y = Ae^{\lambda x}$. Compute the first and second derivatives of this, and then substitute back to get: \[A\lambda^2e^{\lambda x} + 3A\lambda e^{\lambda x} - 10Ae^{\lambda x} = 0 \implies \lambda^{2} + 3\lambda - 10 = 0\]
    So this is our \textbf{auxiliary function}. So we have $\lambda = -5, \lambda = 2$. Our complementary function will be $y = Ae^{-5x} + Be^{2x}$ which is the solution to this differential equation. (This function works as we know each individual term works, and since we are only taking the derivatives, the addition actually doesn't matter, and we can break it down into two parts).

    Depending on the solutions to the auxiliary equation, we would have different complementary functions.

    \begin{theorem}[Roots of auxiliary equation relate to complementary function]
        \[\lambda_1, \lambda_2 \text{ where } \lambda_1 \neq \lambda_2 \implies y = Ae^{\lambda_1 x} + Be^{\lambda_2 x}\]
        \[\lambda_1, \lambda_2 \text{ where } \lambda_1 = \lambda_2 = \lambda \implies y = (Ax + B)e^{\lambda x}\]
        \[\lambda_1, \lambda_2 \text{ where } \lambda_1 = m + ni, \lambda_2 = m-ni \implies y = e^{mx}(A\cos{nx} + B\cos{nx})\]
    \end{theorem}

    In the last case, we would actually have $y = Ce^{(m + ni)x} + De^{(m-ni)x}$. However, if we want $y$ to be real, that will force $C$ and $D$ to be conjugates of each other, which would then derive to give any real values $A$ and $B$ consistent with the third point above.

    \subsection{Second Order: Inhomogeneous}
    When the RHS of the differential equation is no longer $0$, just the complementary function isn't enough. We need to add a $\textbf{particular integral}$.

    \begin{example}
        Find the general solution for \[\frac{d^2y}{dx^2} + 4\frac{dy}{dx} + 4y = 2 + x^2\]
    \end{example}

    If we ignore the RHS we would get $\lambda = -2$ and so $y = (Ax + B)e^{-2x}$. However, since the RHS is a degree 2 polynomial, we would need to add a particular integral in the form of a quadratic. 
    
    Let $y = \alpha x^2 + \beta x + \gamma$. Then we can find first and second derivative, substitute it into the original and compare the coefficients to get that $\alpha = \f{1}{4}, \beta = \frac{-1}{2}, \gamma = \f{7}{8}$. This would give us a general solution of \[y = (Ax + B)e^{-2x} + \frac{1}{4}x^2 - \f{1}{2}x + \f{7}{8}\]


    Now, if the RHS is not a polynomial in $x$, but rather an exponential, we just make the particular integral an exponential as well. However, if the RHS is a trig function such as $\sin{2x}$, then the particular integral needs to be of the form $y = \alpha \sin{2x} + \beta \cos{2x}$ because of the cyclic nature of trig. When we keep differentiating, we the result will be something nice that we can compare coefficients with.

    \subsubsection{Leveling up}
    The order matters. It is important to determine the complementary function first before the particular integral. Sometimes in the edge cases of an exponential, try $\alpha x e^x$ instead of $\alpha e^x$, and if that doesn't work, go one more level up, to try $\alpha x^2 e^x$. The edge case will occur when the RHS matches up with either one or two of the terms in the complementary function.


    \subsection{Substitution}
    If the original differential equation is a function of $\f{y}{x}$ (The form $\dd{y}{x} = F(\f{y}{x})$), then we can use a sub $u = \f{y}{x}$. This will reduce the equation down to a separable form in $u$ and $x$. 

    \begin{example}
        Solve, with a suitable substitution: \[\dd{y}{x} = \f{x - y}{x + y}\]
    \end{example}

    When we substitute $y = ux$ we have $\dd{y}{x} = u + x\dd{u}{x}$. We can also rearrange the original differential equation like so:
    \[\dd{y}{x} = \f{1 - \f{y}{x}}{1 + \f{y}{x}} = \f{1-u}{1+u}\]

    Therefore we have \[u + x\dd{u}{x} = \f{1-u}{1+u}\]

    Which we can separate variables to \[\intg{\f{1+ u}{1 - 2u - u^2}}{u} = \intg{\f{1}{x}}{x}\]

    Which is easy to integrate.

    \vspace{5pt}

    If the function is of the form $\dd{y}{x} = F(ax + by)$, then we can try the substitution $u = ax + by$. 

    \begin{example}
        Find the general solution to \[\dd{y}{x} = e^{4x + 3y}\]
    \end{example}

    Consider the sub $u = 4x + 3y$. Then we know that $\dd{u}{x} = 4 + 3\dd{y}{x}$

    Hence we have \[\frac{\dd{u}{x} - 4}{3} = e^u \iff \intg{\f{1}{4 + 3e^u}}{u} = \intg{1}{x}\]

    We divide the top and bottome by $e^u$, and then we can integrate with $\ln{}$.

    \[-\f{1}{4} \ln{3 + 4e^{-u}} = x + C\]

    Then we just rearrange $u$ back into $x$ and $y$, and we would've found our integral.

    Last type of question would be giving you a differential equation, then telling you to rearrange it to a differential equation in another variable in a specific form. e.g.

    \begin{example}
        Rearrange $x^2 \f{d^2y}{dx^2} + 4x\dd{y}{x} + 2y = \ln{x}$ into $a\f{d^2y}{dt^2} + b \dd{y}{t} + cy = g(t)$ with the substitution $x = e^t$.
    \end{example}

    Notice that $\dd{y}{x} = \dd{y}{t} \times \dd{t}{x}$ by chain rule, so $\dd{y}{x} = \dd{y}{t} \times \f{1}{e^t} = \dd{y}{t} \times \f{1}{x}$. We can do the same thing for the second derivative with the parametric differentiation formula, and then we'll be able to rearrange and solve the desired equation.

    $\textbf{To sum up:}$ Use the substitution they give you, differentiate w.r.t. $x$ or $t$, and then equate the $\dd{y}{x}$ then separate variables/solve differential equation.

\end{document}