%%%%%%%%%%%%%%%%%%%%%%%%%%%%% Define Article %%%%%%%%%%%%%%%%%%%%%%%%%%%%%%%%%%
\documentclass{article}
%%%%%%%%%%%%%%%%%%%%%%%%%%%%%%%%%%%%%%%%%%%%%%%%%%%%%%%%%%%%%%%%%%%%%%%%%%%%%%%

%%%%%%%%%%%%%%%%%%%%%%%%%%%%% Using Packages %%%%%%%%%%%%%%%%%%%%%%%%%%%%%%%%%%
\usepackage{geometry}
\usepackage{graphicx}
\usepackage{amssymb}
\usepackage{amsmath}
\usepackage{amsthm}
\usepackage{empheq}
\usepackage{mdframed}
\usepackage{booktabs}
\usepackage{lipsum}
\usepackage{graphicx}
\usepackage{color}
\usepackage{psfrag}
\usepackage{pgfplots}
\usepackage{bm}
%%%%%%%%%%%%%%%%%%%%%%%%%%%%%%%%%%%%%%%%%%%%%%%%%%%%%%%%%%%%%%%%%%%%%%%%%%%%%%%

% Other Settings

%%%%%%%%%%%%%%%%%%%%%%%%%% Page Setting %%%%%%%%%%%%%%%%%%%%%%%%%%%%%%%%%%%%%%%
\geometry{a4paper}

%%%%%%%%%%%%%%%%%%%%%%%%%% Define some useful colors %%%%%%%%%%%%%%%%%%%%%%%%%%
\definecolor{ocre}{RGB}{243,102,25}
\definecolor{mygray}{RGB}{243,243,244}
\definecolor{deepGreen}{RGB}{26,111,0}
\definecolor{shallowGreen}{RGB}{235,255,255}
\definecolor{deepBlue}{RGB}{61,124,222}
\definecolor{shallowBlue}{RGB}{235,249,255}
%%%%%%%%%%%%%%%%%%%%%%%%%%%%%%%%%%%%%%%%%%%%%%%%%%%%%%%%%%%%%%%%%%%%%%%%%%%%%%%

%%%%%%%%%%%%%%%%%%%%%%%%%% Define an orangebox command %%%%%%%%%%%%%%%%%%%%%%%%
\newcommand\orangebox[1]{\fcolorbox{ocre}{mygray}{\hspace{1em}#1\hspace{1em}}}
%%%%%%%%%%%%%%%%%%%%%%%%%%%%%%%%%%%%%%%%%%%%%%%%%%%%%%%%%%%%%%%%%%%%%%%%%%%%%%%

%%%%%%%%%%%%%%%%%%%%%%%%%%%% English Environments %%%%%%%%%%%%%%%%%%%%%%%%%%%%%
\newtheoremstyle{mytheoremstyle}{3pt}{3pt}{\normalfont}{0cm}{\rmfamily\bfseries}{}{1em}{{\color{black}\thmname{#1}~\thmnumber{#2}}\thmnote{\,--\,#3}}
\newtheoremstyle{myproblemstyle}{3pt}{3pt}{\normalfont}{0cm}{\rmfamily\bfseries}{}{1em}{{\color{black}\thmname{#1}~\thmnumber{#2}}\thmnote{\,--\,#3}}
\theoremstyle{mytheoremstyle}
\newmdtheoremenv[linewidth=1pt,backgroundcolor=shallowGreen,linecolor=deepGreen,leftmargin=0pt,innerleftmargin=20pt,innerrightmargin=20pt,]{theorem}{Theorem}[section]
\theoremstyle{mytheoremstyle}
\newmdtheoremenv[linewidth=1pt,backgroundcolor=shallowBlue,linecolor=deepBlue,leftmargin=0pt,innerleftmargin=20pt,innerrightmargin=20pt,]{definition}{Definition}[section]
\theoremstyle{myproblemstyle}
\newmdtheoremenv[linecolor=black,leftmargin=0pt,innerleftmargin=10pt,innerrightmargin=10pt,]{problem}{Problem}[section]
%%%%%%%%%%%%%%%%%%%%%%%%%%%%%%%%%%%%%%%%%%%%%%%%%%%%%%%%%%%%%%%%%%%%%%%%%%%%%%%

%%%%%%%%%%%%%%%%%%%%%%%%%%%%%%% Plotting Settings %%%%%%%%%%%%%%%%%%%%%%%%%%%%%
\usepgfplotslibrary{colorbrewer}
\pgfplotsset{width=8cm,compat=1.9}
%%%%%%%%%%%%%%%%%%%%%%%%%%%%%%%%%%%%%%%%%%%%%%%%%%%%%%%%%%%%%%%%%%%%%%%%%%%%%%%

%%%%%%%%%%%%%%%%%%%%%%%%%%%%%%% Title & Author %%%%%%%%%%%%%%%%%%%%%%%%%%%%%%%%
\title{9231 CAIE Further Math — Complex Numbers}
\author{Alston}
\date{}
\parindent=0pt
\parskip=5pt
%%%%%%%%%%%%%%%%%%%%%%%%%%%%%%%%%%%%%%%%%%%%%%%%%%%%%%%%%%%%%%%%%%%%%%%%%%%%%%%

\begin{document}
    \maketitle

    \section{Summations}
    We use De Moivre's Theorem to get $z^n = cis(n\theta)$ and also $\frac{1}{z^n} = z^{-n} = cis(-n\theta) = \cos(n\theta) - i\sin(n\theta)$. Then, we can sum or subtract these two statements to get:

    \begin{theorem}[Formulae]
        \[z^n + \frac{1}{z^n} = 2\cos(n\theta)\]
        \[z^n - \frac{1}{z^{n}} = 2i\sin(n\theta)\]
    \end{theorem}

    Now, for evaluating things such as $\int\cos^4(\theta)$, just use the formula for $n=1$ first, then raise it to the 4th power. Expand the bracket in $z$, and then group the $z^n$ terms. 

    \[\cos^4(\theta) = (z + \frac{1}{z})^4 = z^4 + 4z^2 + 6 + 4z^{-2} + z^{-4} = 2\cos(4\theta) + 8\cos(2\theta) + 6\]

    \subsection{Summation of Trig Functions}
    If it's a summation in cos, realise that you can turn it into the REAL parts of the summation of a sequence in $z$, which is very easy to calculate as a geometric series. Then turn it back to $z = cis(\theta)$ or $z = e^{i\theta}$ and get the real part. Similarly for summation in $\sin$.

    \begin{problem}
        Determine the value of $\displaystyle\sum_{n=0}^{\infty}2^{-n}\sin\left(\frac{n\pi}{2}\right)$
    \end{problem}

    Consider the series $\displaystyle\sum_{n = 0}^{\infty}\left(\frac{z}{2}\right)^n$. We actually just want the real part of this series, evaluated at $\theta = \frac{\pi}{2}$. 

    \[\displaystyle\sum_{n = 0}^{\infty}\left(\frac{z}{2}\right)^n = 1 + \frac{z}{2} + \frac{z^2}{4} \cdots = \frac{1}{1 - \frac{z}{2}} = \frac{2}{2 - z}\]

    Now, let $z = e^{i\theta}$, $\frac{2}{2 - z} = \frac{2}{2 - e^{i\theta}}$. Multiply top and bottom by $2 - e^{-i\theta}$ and we have $\frac{2}{2 - z} = \frac{4 - 2e^{-i\theta}}{5 - 4\cos{\theta}}$. Im$\left(\frac{4 - 2e^{-i\theta}}{5 - 4\cos{\theta}}\right) = \frac{2\sin{\theta}}{5 - 4\cos{\theta}}$, so just substitute $\theta = \frac{\pi}{2}$ to get the answer.


    
\end{document}