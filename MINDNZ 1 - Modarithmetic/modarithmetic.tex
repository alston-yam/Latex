
%%%%%%%%%%%%%%%%%%%%%%%%%%%%% Define Article %%%%%%%%%%%%%%%%%%%%%%%%%%%%%%%%%%
\documentclass{article}
%%%%%%%%%%%%%%%%%%%%%%%%%%%%%%%%%%%%%%%%%%%%%%%%%%%%%%%%%%%%%%%%%%%%%%%%%%%%%%%

%%%%%%%%%%%%%%%%%%%%%%%%%%%%% Using Packages %%%%%%%%%%%%%%%%%%%%%%%%%%%%%%%%%%
\usepackage{geometry}
\usepackage{graphicx}
\usepackage{amssymb}
\usepackage{amsmath}
\usepackage{amsthm}
\usepackage{empheq}
\usepackage{mdframed}
\usepackage{booktabs}
\usepackage{lipsum}
\usepackage{graphicx}
\usepackage{color}
\usepackage{psfrag}
\usepackage{pgfplots}
\usepackage{bm}
%%%%%%%%%%%%%%%%%%%%%%%%%%%%%%%%%%%%%%%%%%%%%%%%%%%%%%%%%%%%%%%%%%%%%%%%%%%%%%%

% Other Settings

%%%%%%%%%%%%%%%%%%%%%%%%%% Page Setting %%%%%%%%%%%%%%%%%%%%%%%%%%%%%%%%%%%%%%%
\geometry{a4paper}

%%%%%%%%%%%%%%%%%%%%%%%%%% Define some useful colors %%%%%%%%%%%%%%%%%%%%%%%%%%
\definecolor{ocre}{RGB}{243,102,25}
\definecolor{mygray}{RGB}{243,243,244}
\definecolor{deepGreen}{RGB}{26,111,0}
\definecolor{shallowGreen}{RGB}{235,255,255}
\definecolor{deepBlue}{RGB}{61,124,222}
\definecolor{shallowBlue}{RGB}{235,249,255}
%%%%%%%%%%%%%%%%%%%%%%%%%%%%%%%%%%%%%%%%%%%%%%%%%%%%%%%%%%%%%%%%%%%%%%%%%%%%%%%

%%%%%%%%%%%%%%%%%%%%%%%%%% Define an orangebox command %%%%%%%%%%%%%%%%%%%%%%%%
\newcommand\orangebox[1]{\fcolorbox{ocre}{mygray}{\hspace{1em}#1\hspace{1em}}}
%%%%%%%%%%%%%%%%%%%%%%%%%%%%%%%%%%%%%%%%%%%%%%%%%%%%%%%%%%%%%%%%%%%%%%%%%%%%%%%

%%%%%%%%%%%%%%%%%%%%%%%%%%%% English Environments %%%%%%%%%%%%%%%%%%%%%%%%%%%%%
\newtheoremstyle{mytheoremstyle}{3pt}{3pt}{\normalfont}{0cm}{\rmfamily\bfseries}{}{1em}{{\color{black}\thmname{#1}~\thmnumber{#2}}\thmnote{\,--\,#3}}
\newtheoremstyle{myproblemstyle}{3pt}{3pt}{\normalfont}{0cm}{\rmfamily\bfseries}{}{1em}{{\color{black}\thmname{#1}~\thmnumber{#2}}\thmnote{\,--\,#3}}
\theoremstyle{mytheoremstyle}
\newmdtheoremenv[linewidth=1pt,backgroundcolor=shallowGreen,linecolor=deepGreen,leftmargin=0pt,innerleftmargin=20pt,innerrightmargin=20pt,]{theorem}{Theorem}[section]
\theoremstyle{mytheoremstyle}
\newmdtheoremenv[linewidth=1pt,backgroundcolor=shallowBlue,linecolor=deepBlue,leftmargin=0pt,innerleftmargin=20pt,innerrightmargin=20pt,]{definition}{Definition}[section]
\theoremstyle{myproblemstyle}
\newmdtheoremenv[linecolor=black,leftmargin=0pt,innerleftmargin=10pt,innerrightmargin=10pt,]{problem}{Problem}[section]
\theoremstyle{myproblemstyle}
\newmdtheoremenv[linecolor=black,leftmargin=0pt,innerleftmargin=10pt,innerrightmargin=10pt,]{example}{Example}[section]
\theoremstyle{myproblemstyle}
\newmdtheoremenv[linecolor=black,leftmargin=0pt,innerleftmargin=10pt,innerrightmargin=10pt,]{exercise}{Exercise}[section]
%%%%%%%%%%%%%%%%%%%%%%%%%%%%%%%%%%%%%%%%%%%%%%%%%%%%%%%%%%%%%%%%%%%%%%%%%%%%%%%

%%%%%%%%%%%%%%%%%%%%%%%%%%%%%%% Plotting Settings %%%%%%%%%%%%%%%%%%%%%%%%%%%%%
\usepgfplotslibrary{colorbrewer}
\pgfplotsset{width=8cm,compat=1.9}
\parindent=0in
%%%%%%%%%%%%%%%%%%%%%%%%%%%%%%%%%%%%%%%%%%%%%%%%%%%%%%%%%%%%%%%%%%%%%%%%%%%%%%%

%%%%%%%%%%%%%%%%%%%%%%%%%%%%%%% Title & Author %%%%%%%%%%%%%%%%%%%%%%%%%%%%%%%%
\title{Molular Arithmetic \& Inverses}
\date{}
\author{Alston Yam}
%%%%%%%%%%%%%%%%%%%%%%%%%%%%%%%%%%%%%%%%%%%%%%%%%%%%%%%%%%%%%%%%%%%%%%%%%%%%%%%

\begin{document}
    \maketitle
    \section{Preview/Review of our class!}
    \subsection{Modular Arithmetic}
    \begin{definition}
        We write $$a \equiv b \pmod{n}$$ if and only if $n \mid a - b$.
    \end{definition}
    Notice that, with this handy notation, we have  $$a = qn + r \iff a \equiv r \pmod{n}$$ This means that this modular arithmetic notation is closely related to the division algorithm and remainders.
    
    \begin{exercise}
        Evaluate $86 \equiv \text{ ?} \pmod{17}$
    \end{exercise}
    \begin{exercise}
        Evaluate $91 \equiv \text{ ?} \pmod{13}$
    \end{exercise}

    \vspace{3pt}
    The power of modular arithmetic actually extends beyond just remainders! As we will soon see, our addition, subtraction, multiplication, and division operations still (mostly) hold true in modular arithmetic.

    \subsubsection{Addition, Subtraction, and Multiplication}
    The operations of addition, subtraction, and multiplication are compatible with modular arithmetic. Specifically, if $a \equiv b \pmod{n}$ and $c \equiv d \pmod{n}$, then:
    \begin{align*}
        a + c &\equiv b + d \pmod{n} \\
        a - c &\equiv b - d \pmod{n} \\
        ac &\equiv bd \pmod{n}
    \end{align*}

    \begin{exercise}
        By writing $a = An + r_1, b = Bn + r_1, c = Cn + r_2, d = Dn + r_2$, prove each of the above statements.
    \end{exercise}

    \subsubsection{Division}
    Unlike the three other operations we have seen, division is not always compatible with modular arithmetic. See below: 
    \begin{example}
        $$6 \equiv 0 \pmod{3}$$ but $$\frac{6}{3} = 2 \not\equiv 0 = \frac{0}{3} \pmod{3}$$
    \end{example}
    So, to handle division, we must introduce the concept of \textbf{inverses}.

    \subsection{Inverses}
    \begin{definition}[Inverses]
        Let $a$ be an integer and $n$ a positive integer. We say that $x$ is the \textbf{inverse} of $a$ modulo $n$ if $$a \cdot x \equiv 1 \pmod{n}$$
        Here, we may write $x \equiv a^{-1} \pmod{n}$.
    \end{definition}

    Inverses \textbf{do not} always exist. 
    
    \begin{example}
        if $n = 6$, then $2$ does not have an inverse modulo $6$ because $$2 \cdot x \equiv 1 \pmod{6}$$ has no solution.
    \end{example}
    
    \vspace{3pt}
    So instead of jumping to the general $n$, lets first investigate the special case where $n=p$ is a prime. 
    \vspace{3pt}
    Consider the set of integers $\{1, 2, \ldots, p-1\}$, and multiply each element by an integer $a$ coprime to $p$ (This will become important later). So we arrive at the set $\{a, 2a, \ldots, (p-1)a\}$. I claim that:
    \vspace{5pt}

    \textbf{Lemma:} The second set is actually a permutation of the first under mod $p$. 
    
    \begin{proof}
    We will go by arguing that every element in the second set is distinct under mod $p$. This is sufficient, as we can see that if no two elements are the same under mod $p$, then the $p-1$ elements in the second set must each match up to one of the $p-1$ elements in the first set.
    
    So suppose $\exists \text{ }i, j$ such that $ia \equiv ja \pmod{p}$ for some $1 \leq i < j \leq p-1$. Then we have $$ia - ja \equiv 0 \pmod{p} \iff (i-j)a \equiv 0 \pmod{p} \iff p \mid a(i - j)$$ Since $a$ is coprime to $p$, we must have that $p \mid i - j$, which implies that $i = j$, a contradiction to $i < j$. Thus, every element in the second set is distinct under mod $p$, and we're done.
    \end{proof}


    \vspace{5pt}
    What does this mean? Well, for any $a$ coprime to $p$, we can actually find an $x \in \{1, 2, \ldots, p-1\}$ such that $$ax \equiv 1 \pmod{p}$$ This means that every integer $a$ coprime to $p$ has an inverse modulo $p$, which is a really cool result!
    \vspace{5pt}

    Now lets think about how we can generalise this to any positive integer $n$. 
    \begin{exercise}
        Prove that for an integer $n$, every integer $a$ coprime to $n$ has an inverse modulo $n$.
    \end{exercise}
    Hint: think about where did we use the fact that $a$ was coprime to $p$ in our previous proof? Does our proof still hold true even if $n$ is now no longer a prime?
    
    \vspace{5pt}
    Let's turn our attention back to our goal: dividing in modular arithmetic.
    \begin{example}
        Solve for $x$ in $$ax \equiv b \pmod{p}$$ 
    \end{example}
    \textbf{Solution:} $$x \equiv ax \cdot a^{-1} \equiv b \cdot a^{-1} \pmod{p}$$

    As we can see, we have essentially ``divided" both sides by $a$. In fact, if we write $a^{-1} \equiv \frac{1}{a} \pmod{p}$, we have the property that inverses will behave just like fractions.
    \begin{example}
        Show that $$\frac{a}{b} + \frac{c}{d}\equiv \frac{ad + bc}{bd} \pmod{p}$$  
    \end{example}
    \textbf{Solution:} $$ab^{-1} + cd^{-1} \equiv (ad + bc)(bd)^{-1} \equiv \frac{ad + bc}{bd} \pmod{p}$$
    \begin{exercise}
        In a similar fasion, show that $$\frac{a}{b} \cdot \frac{c}{d}\equiv \frac{ac}{bd} \pmod{p}$$  
    \end{exercise}

    \subsection{A Couple of Useful Results}
    \begin{theorem}[Fermat's Little Theorem]
        Let $a$ be any integer relatively prime to a prime $p$. Then $$a^{p-1} \equiv 1 \pmod{p}$$
    \end{theorem}

    \begin{theorem}[Euler's Totient Theorem]
        If $gcd(a, m) = 1$, then we have $$a^{\phi(m)} \equiv 1 \pmod{m}$$ Where $\phi(m)$ represents the number of positive integers $k \leq m$ with $gcd(k, m) = 1$.
    \end{theorem}
    Keen eyed readers might notice that Euler's Totient Theorem is in fact a generalisation of \\ Fermat's Little Theorem!

    \begin{theorem}[Wilson's Theorem]
        Let $p$ be a prime. Then $$(p-1)! \equiv -1 \pmod{p}$$
    \end{theorem}

    \pagebreak

    \section{Problems}
        \begin{problem}
            Prove that $15m - 3 \equiv 0 \pmod{7}$ if and only if $41m - 10 \equiv 0 \pmod{7}$
        \end{problem}
        \begin{problem}
            Is $3^{16} - 2^{16}$ divisible by $17$?
        \end{problem}

        \begin{problem}
            Let $p$ be a prime. Prove that if $a$ is an integer such that $gcd(a, p) = 1$, then $a^{p-2} \equiv a^{-1} \pmod{p}$.   
        \end{problem}

        \begin{problem}
            Let $p$ be a prime, show that if $p$ divides $a^p - 1$ then $p^2$ divides $a^p-1$.
        \end{problem}

        \begin{problem}
            Prove each of the three theorems in section 1.3.
        \end{problem}

        \begin{problem}[St. Petersburg 2008]
            Given three distinct natural numbers $a, b, c$, show that $$gcd(ab+1, bc + 1, ca + 1) \leq \frac{a+b+c}{3}$$
        \end{problem}

        \begin{problem}[IMO 2005 Q4]
            Determine all positive integers relatively prime to all the terms of the infinite sequence $$a_n = 2^n + 3^n + 6^n - 1, n \geq 1$$
        \end{problem}
\end{document}